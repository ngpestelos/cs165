\documentclass[12pt]{article}
\usepackage{longtable}
\textwidth=7in
\textheight=9.5in
\topmargin=-1in
\headheight=0in
\headsep=.5in
\hoffset  -.85in

\pagestyle{empty}

\renewcommand{\thefootnote}{\fnsymbol{footnote}}
\begin{document}

\begin{center}
{\bf CS165 Database Systems  
\\ Homework 1 - ER Modeling
}
\end{center}

\setlength{\unitlength}{1in}

\begin{picture}(6,.1) 
\put(0,0) {\line(1,0){6.25}}         
\end{picture}

\vskip.25in
\noindent\textbf{Submission:} November 23, Class

\vskip.25in
\noindent \textbf{Requirements}:
\begin{itemize}
  \item ER Diagram
  \item Indicate in footnotes any covering or overlap constraints for classes.
  \item (Optional) Short document explaining any decisions made in constructing your diagram.
\end{itemize}

\vskip.25in
\noindent \textbf{Problem}:
\vskip.15in
\noindent Your school organization asked you to write a database to manage the data that is used by your organization. They did not give you any formal specifications, but by reading their constitution,  you have deduced the following requirements:

\begin{itemize}
  \item A student has a unique student number and a name.
  \item Students specialize in at most one field (e.g. Computer Science, Biology, etc.) Every field of specialization has a unique name.
  \item All students must provide at least one residence address for the database. An address has a number, street, city/municipality, and a province. No two addresses have the same values in all fields. An address may not be exclusive to one student. 
  \item Some of the addresses have phone numbers. A phone number has a unique value. Each phone number pertains to exactly one address.
  \item All students must provide at least one mobile phone number. A mobile phone number has a provider (e.g. Globe, Smart) and a unique value. A mobile phone number may not be exclusive to one student.
  \item All students must provide at least one email address. An email address has a unique value. An e-mail addresses may not be exclusive to one student.
  \item A student is either an applicant, or a member; never both at any given time.
  \item All members are required to enter their GWA into the database.
  \item An applicant must have exactly one member as a buddy.
  \item A member can have zero or more applicants as buddies.
  \item Each member must work in exactly one committee. A committee can be uniquely identified by a committee name (records, finance, logistics, etc.). There must be at least one member working for each committee. We are interested with the starting date of a member working for a committee.
  \item A committee must also have exactly one member as the head of the committee. We are interested with the starting date of a member heading a committee.
  \item Each applicant must work as an intern in exactly one committee. Every internship must be monitored by at least one member.
  \item Aside from committees, the organization also has teams for organizing events and activities. A team can be identified uniquely by name. There must be at least one member working for each team.
  \item Each team can organize an activity. An activity has a name. Some activities can have the same name as long as they are organized by different teams. If a team is deleted from the database, its activities must also be removed.
\end{itemize}

\noindent Design an ER diagram that captures the requirements. Use the basic ER model (entities, relationships, and attributes). Indicate key and participation constraints. Also indicate overlap and covering constraints.

\end{document}