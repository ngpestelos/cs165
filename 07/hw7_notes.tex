\documentclass[10pt]{article}
\textwidth=7in
\textheight=9.5in
\topmargin=-1in
\headheight=0in
\headsep=.5in
\hoffset=-.85in
\pagestyle{empty}
\usepackage{amsmath}
\usepackage{hyperref}

\hypersetup{colorlinks=true}

\begin{document}

\begin{center}
{\bf CS165 Database Systems \\
Secondary Storage Management (Summary)}
\end{center}

\vskip.35in
\noindent Databases almost always involve working with disks (secondary storage). For this lesson, we seek answers to these questions: \\

\begin{itemize}
\item What are the types of storage (and how fast are each one)?
\item How do disks work?
\item What factors affect the speed of accessing data?
\item What are the types of disk failure (and what can be done about them)?
\item How is data stored in disk (and how are they accessed)?
\end{itemize}

\vskip.15in
\noindent {\bf What are the types of storage?}

\vskip.15in
\noindent Think of this as a hierarchy, where the fastest {\it CPU cache} is at the top, followed by {\it main memory}, then {\it secondary storage}. Each type of storage is an order of magnitude faster (or slower). \\

\noindent Memory can be {\it volatile} or {\it non-volatile}. Storage that is non-volatile (e.g. hard drives) can keep data even after the power supplied to a machine is turned off. Cache memory and main memory are volatile.

\vskip.15in
\noindent {\bf How do disks work?}

\vskip.15in
\noindent Data are stored in {\it blocks}, which are stored in a {\it sector} of a {\it track}. For simplicity, we limit our discussion to conventional hard drives (rotating platters) and not SSDs.\\

\noindent A disk consists of {\it platters}, {\it heads}, and a {\it controller}. A {\it cylinder} is the set of tracks that are currently under the disk head. This configuration will play a role in computing the blocks' physical addresses.\\

\noindent The disk controller is a dedicated circuit responsible for (1) moving the disk heads, (2) selecting the appropriate sector to transfer data, (3) perform the actual data transfer, (4) buffer tracks to main memory. \\

\noindent Suppose we have a disk with the following specifications (assuming no gaps between sectors):

\begin{itemize}
\item 8 platters (16 surfaces)
\item 65,536 tracks per surface
\item 256 sectors per track
\item 4,096 bytes per sector
\end{itemize}

\noindent To compute the disk's capacity (1 TB), we multiply the number of surfaces with the number of tracks ({\it 1,048,576} tracks), then times the number of sectors per track ({\it 268,435,546} sectors), and finally, multiplied with the number of bytes per sector ({\it 1,099,511,627,776} bytes).

\newpage

\noindent {\bf What factors affect the speed of accessing data?}

\vskip.15in
\noindent Compared to main memory, disks are slow. Accessing data involves three steps: (1) position the head at the right track ({\it seek time}), (2) wait until the desired sector falls under the head ({\it rotational delay}), and (3) read/write data in the sector ({\it block access time}). \\

\noindent Using the disk used in the previous example (re: computing capacity) and suppose we now have the following information and we are tasked to compute the minimum, maximum, and average times for a 16,384-byte block:

\begin{itemize}
\item rotates at 7,200 rpm (0.0083 ms per rotation)
\item heads move at 1.00025 milliseconds per track
\item gaps occupy 10 percent of the space around the track
\end{itemize}

\noindent The minimum time is simply the block transfer time (head is already at the desired track, with the block's first sector already below the head). \\

\noindent The average time is the ratio of how long it takes for the head to span the block versus the time it takes to span the whole track. \\

\noindent ({\it Scheduling latency}) It is not always the case that if a disk is rated at 10 milliseconds that it is able to perform at that speed. {\it Throughput} is the number of requests processed per second. Increasing throughput could involve one or more of these steps:

\begin{itemize}
\item Place related blocks together in the same cylinder.
\item Distribute data across multiple disks. This increases the block access per unit time.
\item Mirror a disk to access more blocks at the same time.
\item Use a disk-scheduling algorithm to efficiently order how blocks are accessed.
\item Buffer frequently used blocks in main memory.
\end{itemize}

\vskip.15in
\noindent {\bf What are the types of disk failure (and what can be done about them)?} \\

\noindent Part of the database's job is to ensure data is not lost. Failure could happen when:

\begin{itemize}
\item An attempt to read or write a sector is unsuccessful ({\it intermittent failure}).
\item Bits are permanently corrupted ({\it media failure}).
\item An error occured while writing a block to disk and the previous data no longer can be retrieved ({\it write failure}).
\item The entire disk becomes unreadable, possibly permanently {\it disk crash}.
\end{itemize}

\noindent The disk controller can use checksums to check if the sector has been correctly read. In case it needs to write a sector, but the sector's integrity is unknown, it needs to read the data again. A checksum is a set of bits that tells if the sector has been properly written to disk. \\

\noindent But how can we correct the error? One approach is to store another copy of the sector to be written ({\it stable-storage policy}). \\

\noindent As mentioned previously when speeding up data access, we can also consider using more than one disk. \\

\newpage

\noindent {\bf How is data stored in disk (and how are they accessed)?} \\

\noindent A {\it record} contains a set of {\it fields}. Think of records as similar to {\it structs} in C. A record can be of fixed length or have a variable length. Each record has a {\it header} containing {\it metadata} such as length and pointers to its fields. \\

\noindent Blocks store a collection of records and similar to a record, it has its own header. It also has a pointer to the next block (note: this will be discussed further in the Indexing lesson). \\

\noindent Databases refer to blocks using addresses. There are two types of addresses: physical and database addresses. When moving blocks to and from disk, an address translation is performed ({\it pointer swizzling}). \\

\noindent Inserting a record to a block, for the simplest case, involves adding a new record to the block. However, if the block must only contain a fixed number of records, then an {\it overflow block} is created. \\

\noindent Deleting a record provides an opportunity to reclaim space. However, we would want to avoid {\it dangling pointers} (i.e., pointers to deleted records). A special record called a {\it tombstone}, indicates that the record has been deleted. \\

\noindent Updating a fixed-length record is straightforward, as it occupies the same location in the block. However, when working with a variable-length record, the block may need to be resized (shrunk or grown) to fit the updated record, or an overflow block could be created.

\vskip.15in
\noindent {\bf See Also}

\begin{itemize}
\item \href{http://goo.gl/4FccW}{What are numbers every computer engineer should know?}
\item \href{http://goo.gl/3tn5xv}{The elevator algorithm}
\item \href{http://en.wikipedia.org/wiki/RAID}{Redundant Array of Inexpensive Disks}
\item \href{http://en.wikipedia.org/wiki/Ssd}{Solid-State Drive}
\end{itemize}

\end{document}