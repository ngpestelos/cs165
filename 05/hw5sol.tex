\documentclass[12pt]{article}
\textwidth=7in
\textheight=9.5in
\topmargin=-1in
\headheight=0in
\headsep=.5in
\hoffset=-.85in
\pagestyle{empty}
\usepackage{amsmath}

\begin{document}

\begin{center}
{\bf CS 165 Database Systems
\\ Solutions to Homework 5 - Database Normalization}
\end{center}

\setlength{\unitlength}{1in}

\noindent 1. Consider the relation

\[Properties(name, year, sqm, city, radius, zipcode)\]

\vskip.15in
\noindent A property, acquired at some year, of a certain size (in square meters) is located within some radius from a city having a unique zipcode. Also, a property can only belong to one city.

\vskip.15in
\noindent What functional dependencies (FDs) do you expect to hold?

\vskip.15in
\noindent {\bf Solution:}

\vskip.15in
\noindent A functional dependency means that tuples $t,u \hspace{.05in} \epsilon \hspace{0.05in} R$

\[ t_{a_1,a_2,..,a_n} = u_{a_1,a_2,..,a_n} \Rightarrow t_{b_1,b_2,..,b_n} = u_{b_1,b_2,..,b_n}\]

\begin{equation} {name, year, city} \rightarrow {sqm, radius} \end{equation}
\vskip.15in
\begin{equation} {city} \rightarrow {zipcode} \end{equation}

\vskip.45in
\noindent 2. Consider the relation

\[R(A, B, C, D)\]

\vskip.15in
\noindent with FDs

\[AB \rightarrow C, C \rightarrow D, D \rightarrow A\]

\vskip.15in
\noindent What are all the nontrivial FDs that follow from the given FDs?
\vskip.15in
\noindent\textbf{Solution:}

\vskip.15in
\noindent A functional dependency ($X \rightarrow Y$) is nontrivial if $Y \not\subset X$. Using the transitive rule, we find the following nontrivial FDs:

\[AB \rightarrow D, C \rightarrow A \]

\vskip.15in
\noindent What are all the keys of $R$?

\vskip.15in
\noindent A key is defined as the set of attributes that functionally determines all of the attributes in a  relation. For this question, we can use the closure of one of the FDs to help us find a key:

\[\{A,B\}^+ = \{A,B,C,D\} \]

\vskip.15in
\noindent 3. Consider the relation

\small{\[R(custID, custName, street, city, email, title, year, actor, genre, rentID, rentPrice, rentDate, rentLimit)\]}

\noindent Decompose this relation into smaller relations such that no data is lost when the decomposed relations are joined. The goal of this exercise is to keep redundancies and inconsistencies to a minimum.

\end{document}