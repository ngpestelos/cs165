\documentclass[12pt]{article}
\textwidth=7in
\textheight=9.5in
\topmargin=-1in
\headheight=0in
\headsep=.5in
\hoffset=-.85in
\pagestyle{empty}
\usepackage{amsmath}

\begin{document}

\begin{center}
{\bf CS165 Database Systems
\\ Homework 5 - Normalization
\\ Submission: January 25, 2014 (class)}
\end{center}

\setlength{\unitlength}{1in}

\noindent [2 points] 1. Consider the relation

\[Properties(name, year, sqm, city, radius, zipcode)\]

\vskip.15in
\noindent A property, acquired at some year, of a certain size (in square meters) is located within some radius from a city having a unique zipcode. Also, a property can only belong to one city.

\vskip.15in
\noindent What functional dependencies (FDs) do you expect to hold?

\vskip.45in
\noindent [2 points] 2. Consider the relation

\[R(A, B, C, D)\]

\vskip.15in
\noindent with FDs

\[AB \rightarrow C, C \rightarrow D, D \rightarrow A\]

\vskip.15in
\noindent What are all the nontrivial FDs that follow from the given FDs? Limit your answers to one attribute on the right-hand side.
\vskip.15in
\noindent What are all the keys of $R$?
\vskip.15in

\vskip.45in
\noindent [6 points] 3. Consider a movie rental relation

\small{\[R(custID, custName, street, city, email, title, year, actor, genre, rentID, rentPrice, rentDate, rentLimit)\]}

\noindent Decompose this relation into smaller relations such that no data is lost when the decomposed relations are joined. The goal of this exercise is to keep redundancies and inconsistencies to a minimum.

\end{document}