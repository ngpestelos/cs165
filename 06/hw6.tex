\documentclass[12pt]{article}
\textwidth=7in
\textheight=9.5in
\topmargin=-1in
\headheight=0in
\headsep=.5in
\hoffset=-.85in
\pagestyle{empty}
\usepackage{amsmath}

\begin{document}

\begin{center}
{\bf CS165 Database Systems
\\ Homework 6 - Concurrency Control
\\ Submission: February 8, 2014 (class)}
\end{center}

\setlength{\unitlength}{1in}

\noindent
1. [3 points] A transaction $T_1$, executed by an airline reservation system, performs the following steps:

\begin{quote}
$i.$ It searches for flights and is able to load records $A$ and $B$ from the database. \\
$ii.$ Customer selects flight $B$ and a reservation is made for that customer. \\
$iii.$ Customer selects a seat for the flight, denoted by record $C$. \\
$iv.$ It fetches the customer's credit card data and sends a bill for the flight ($D$). \\
$v.$ Customer's rewards data and the flight's mileage are also loaded ($E$ and $F$). The customer's rewards data is then updated.
\end{quote}

\vskip.05in
\noindent Express $T_1$ as as series of $r$ and $w$ actions.

\vskip.35in
\noindent 2. [3 points] For the following schedule:

\begin{quote}
$r_1(A); r_2(A); r_3(B); w_1(A); r_2(C); r_2(B); w_2(B); w_1(C)$
\end{quote}

\vskip.10in
\noindent Answer the following questions:

\begin{quote}
$i.$ What is the precedence graph of the schedule? \\
$ii.$ Is the schedule conflict-serializable? If so, give one equivalent serial schedule.
\end{quote}

\vskip.35in
\noindent 3. [4 points] For the following schedule:

\begin{quote}
$r_1(A); r_2(B); r_3(C); w_1(B); w_2(C); w_3(D);$
\end{quote}

\vskip.10in
\noindent Insert shared and exclusive locks, and insert unlock actions. Place a shared lock ($sl_i(X)$) immediately in front of each read action that is not followed by a write action of the same element of the same transaction. Place an exclusive lock ($xl_i(X)$) in front of every other read or write action. Place the necessary unlocks at the end of every transaction.

\end{document}